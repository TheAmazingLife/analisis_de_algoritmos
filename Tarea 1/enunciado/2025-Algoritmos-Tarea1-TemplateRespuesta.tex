\documentclass{article}

\usepackage{hyperref}
\usepackage{listings}
\usepackage{xcolor}
\lstset { %
    language=C++,
    backgroundcolor=\color{black!5}, % set backgroundcolor
    basicstyle=\footnotesize,% basic font setting
}
\begin{document}
\title{Analisis de Algoritmos \\ Tarea 1}
%\date{18 de Abril 2025  [2025-04-18 Fri]} 
\author{Nombres de los alumnos del grupo}

\maketitle


\section{Fuerza Bruta}

\subsection{Implementación}
  Implemente un algoritmo \texttt{brute\_force} calculando las $n(n-1)$ distancias de manera a seleccionar la distancia mínima.  
\begin{lstlisting}
\end{lstlisting}

\subsection{Correctitud}
Realice un análisis de correctitud del algoritmo.

\subsection{Complejidad Computacional}
Defina y demuestre su complejidad computacional. 
  
\section{Dividir para Vencer}

\subsection{Diseño}

Diseñe o busque en Internet (cite su fuente) un algoritmo \texttt{divide\_and\_conquer} usando dividir para vencer para encontrar la distancia mínima en tiempo en $o(n^2)$.

\subsection{Correctitud}
Realice un análisis de correctitud del algoritmo.

\subsection{Complejidad Computacional}
Defina y demuestre su complejidad computacional. 


  
\section{Implementación}

Implemente su algoritmo \texttt{divide\_and\_conquer}, comparando sus respuestas con las de su implementación de \texttt{brute\_force}.

  \begin{lstlisting}
\end{lstlisting}

  
  \section{Análisis Experimental}

  \subsection{Diseño}

  Diseñe un análisis experimental calculando los tiempos de ejecución en nanosegundos de sus dos soluciones para un conjunto de $n$ puntos cuyas coordenadas enteras están elegidas al azar en un cuadrado de $100\times100$, variando la cantidad $n$ de puntos entre las potencias de dos de $2^3=8$ a $2^{9}=512$.

  \subsection{Realización}

Realice el análisis experimental diseñado, guardando los resultados obtenidos.

  
  \begin{lstlisting}
\end{lstlisting}
    
\section{Mejora} Observando el tiempo de ejecución para valores grandes de $n$, proponga y evalúe una versión trivialmente mejorada de los dos algoritmos.


\section{Gráfico} Construya un gráfico que muestre cómo varían los tiempos de ejecución de sus cuatros soluciones en nanosegundos, variando la cantidad $n$ de puntos entre las potencias de dos de $2^3=8$ a $2^9=512$.

%  \begin{figure}
%  \centering
%  \includegraphics[width=\linewidth]{Programming/MinDistance/plot}
%  \caption{Tiempo de ejecución de de los 4 algoritmos implementados.}\label{fig:fourAlgorithms}
%  \end{figure}




\end{document}

%%% Local Variables:
%%% mode: latex
%%% TeX-master: t
%%% End:
